\chapter{Risultati}
% Tabelle e grafici di ogni esperimento, con confronti fra i continual
\section{Risultati finali}
\paragraph{WESAD} Usando come modello due layer GRU da 18 unità, su WESAD sono stati ottenuti i seguenti risultati:

\begin{table}[h]
\footnotesize
    \begin{tabular}{l|c|c|c|c|c|c|c}
        \textbf{Scenario} & \textbf{Epoche} & \textbf{Tempo} & \textbf{Acc.} & \textbf{ACC} & \textbf{BWT} & \textbf{FWT} & \textbf{Memoria}\\
        \hline
        \textbf{Offline} & 28 & 94,69s & 99,07 & - & - & - & 2061,40 Mb\\
        \textbf{Continual} & 49,14$\pm$33,67 & 947,24s & 73,13$\pm$4,02 & 0,7721 & 0,0343 & 0,5397 & 2173 Mb\\
        \textbf{Cumulative} & 39,71$\pm$19,91 & 2786s & 81,97$\pm$8,67 & 0,961 & 0,1383 & 0,4674 & 2291,45 Mb\\
        \textbf{Replay} & 41,14$\pm$21,19 & 1063,51s & 78,29$\pm$3,32 & 0,7849 & -0,002 & 0,4582 & 2184,77 Mb\\
        \textbf{Episodic} & 35$\pm$26,26 & 1088,29s & 82,13$\pm$6,60 & 0,9095 & 0,0841 & 0,4226 & 2097,47 Mb\\
        \textbf{EWC} & 29,71$\pm$17,38 & 1342,81s & 70,74$\pm$4,74 & 0,7251 & 0,0113 & 0,4698 & 2187,40 Mb\\
        \textbf{LWF} & 44,29$\pm$18,30 & 3282,51s & 69,09$\pm$5,82 & 0,7419 & 0,0451 & 0,3248 & 2121,31 Mb\\
    \end{tabular}
    \caption{Risultati WESAD}
    \label{tab:reswesad}
\end{table}
\begin{figure}[h]
	\begin{center}
		\includegraphics[width=0.95\textwidth]{img/graphs/wesad_final_metrics.png}
		\caption{Risultati su WESAD}
		\label{fig:wesad_metrics_graph}
	\end{center}
\end{figure}

L'accuratezza del 99\% raggiunta nel training offline dimostra come il dataset WESAD sia particolarmente adatto all'inferenza da parte di reti neurali ricorrenti dalla bassa complessità. Con poco più di 25 epoche di addestramento, abbiamo ottenuto un modello capace di classificare con un alto grado di accuratezza lo stato psico-fisico di un utente.\\
Ciò che è interessante notare è che l'accuratezza è mantenuta sopra il 70\% anche nella maggior parte degli approcci continual. La performance peggiore media la si ha nel \textit{Learning Without Forgetting}, che però ottiene comunque un'accuratezza finale (ACC) del 74\% e un FWT comunque ottimo. La miglior performance tra gli approcci continual è ottenuta dall'apprendimento cumulativo, con un'accuratezza finale del 96\% perfettamente paragonabile all'approccio offline, e in linea con la teoria. In tutti gli approcci, si ha un FWT elevato e molto superiore al BWT, segnale che ogni soggetto migliora molto l'apprendimento sui soggetti successivi, mentre non vi è particolare influenza sui precedenti soggetti o, se vi è, è positiva.\\\\
L'esperimento quindi dimostra come l'apprendimento continuo su dati sensoriali come respirazione, temperatura, elettrocardiogramma e simili possa portare a modelli con performance paragonabili all'apprendimento offline. Con un po' di \textit{fine-tuning} e \textit{model selection} con gli approcci continui si può probabilmente trovare un modello con performance ancora superiori a quelle sperimentate.
\begin{figure}[h]
	\begin{center}
		\includegraphics[width=0.65\textwidth]{img/graphs/wesad_final_accuracy.png}
		\caption{Accuratezza su WESAD}
		\label{fig:wesad_accuracy_graph}
	\end{center}
\end{figure}
\pagebreak
\paragraph{ASCERTAIN} Usando come modello due layer GRU da 24 unità, su ASCERTAIN sono stati raggiunti i seguenti risultati:
\begin{table}[h]
\footnotesize
    \begin{tabular}{l|c|c|c|c|c|c|c}
        \textbf{Scenario} & \textbf{Epoche} & \textbf{Tempo} & \textbf{Acc.} & \textbf{ACC} & \textbf{BWT} & \textbf{FWT} & \textbf{Memoria}\\
        \hline
         \textbf{Offline} & 3 & 29,14s & 42,78 & - & - & - & 1817,28 Mb\\
        \textbf{Continual} & 10,88$\pm$5,01 & 249,28s & 37,45$\pm$5,01 & 0,25 & -0,0168 & 0,0213 & 2154,36 Mb\\
        \textbf{Cumulative} & 9,25$\pm$6,81 & 932,56s & 39,06$\pm$4,26 & 0,2697 & 0,0064 & 0,0278 & 2297,17 Mb\\
        \textbf{Replay} & 13,25$\pm$10,03 & 402,96s & 39,48$\pm$4,37 & 0,2603 & -0,014 & 0,0485 & 2173,95 Mb\\
        \textbf{Episodic} & 12,62$\pm$7,94 & 498,24s & 38,80$\pm$4,04 & 0,2742 & 0,0048 & 0,0156 & 2231,42 Mb\\
        \textbf{EWC} & 24,14$\pm$16,94 & 810,96s & 36,08$\pm$5,66 & 0,2497 & -0,0183 & -0,0003 & 2171,62 Mb\\
        \textbf{LWF} & 21,62$\pm$10,20 & 879,68s & 35,69$\pm$5,43 & 0,2448 & 0.0327 & 0,0156 & 2103 Mb\\
    \end{tabular}
    \caption{Risultati ASCERTAIN}
    \label{tab:resascertain}
\end{table}
\begin{figure}[h]
	\begin{center}
		\includegraphics[width=0.95\textwidth]{img/graphs/ascertain_final_metrics.png}
		\caption{Risultati sul dataset ASCERTAIN}
		\label{fig:ascertain_metrics_graph}
	\end{center}
\end{figure}

Dall'approccio offline è subito chiaro come il dataset ASCERTAIN presenti una difficoltà maggiore rispetto a WESAD. In particolare, il dataset ASCERTAIN presenta dati particolarmente sbilanciati sulla classe 0 e informazioni riguardo le espressioni facciali dei soggetti durante l'esperimento. Questi dati sono di difficile classificazione.\\
Gli approcci continui sono comunque in linea con i risultati sul precedente dataset, con la performance peggiore ottenuta dal \textit{Learning Without Forgetting} mentre la migliore è ottenuta dall'approccio cumulativo grazie ad un BWT superiore all'approccio di replay. In molti casi si nota un trasferimento negativo di conoscenza all'indietro, sintomo di un lieve caso di \textit{catastrophic forgetting}.\\\\
Per poter dimostrare l'impatto dovuto allo sbilanciamento delle classi, il dataset ASCERTAIN è stato riorganizzato in soggetti "fittizi" come esposto di seguito.
\pagebreak
\paragraph{Custom ASCERTAIN} Dopo aver eseguito il preprocessing specificato precedentemente, tutti i dati di addestramento sono stati uniti in un unico insieme. Da questo insieme è stato estratto il test set, e il rimanente training set è stato suddiviso in 17 soggetti bilanciati da 108 sequenze di 160 punti ciascuno.\\
Usando come modello due layer GRU da 24 unità, su ASCERTAIN con i soggetti fittizi sono stati registrati i seguenti risultati:
\begin{table}[h]
\footnotesize
    \begin{tabular}{l|c|c|c|c|c|c|c}
        \textbf{Scenario} & \textbf{Epoche} & \textbf{Tempo} & \textbf{Acc.} & \textbf{ACC} & \textbf{BWT} & \textbf{FWT} & \textbf{Memoria}\\
        \hline
         \textbf{Offline} & 29 & 66,17s & 87,77 & - & - & - & 1945,91 Mb\\
        \textbf{Continual} & 7,75$\pm$5,33 & 226,24s & 87,01$\pm$0,64 & 0,2481 & -0,0241 & 0,0052 & 2143,71 Mb\\
        \textbf{Cumulative} & 7,25$\pm$8,80 & 872,8s & 87,66$\pm$0,16 & 0,2773 & 0,0032 & 0,0801 & 2273,07 Mb\\
        \textbf{Replay} & 12,125$\pm$12,94 & 299,52s & 87,13$\pm$0,75 & 0,2711 & -0,0012 & 0,306 & 2146,35 Mb\\
        \textbf{Episodic} & 5,5$\pm$3,94 & 319,76s & 82,19$\pm$3,95 & 0,339 & 0,0346 & 0,1043 & 2204,84 Mb\\
        \textbf{EWC} & 14,38$\pm$7,58 & 541,68s & 87,34$\pm$0,42 & 0,2467 & -0,0191 & 0,0845 & 2146,36 Mb\\
        \textbf{LWF} & 20,75$\pm$8,42 & 771,36s & 86,91$\pm$0,89 & 0,2495 & -0,0333 & -0,146 & 2095,93 Mb\\
    \end{tabular}
    \caption{Risultati su ASCERTAIN con soggetti fittizi bilanciati}
    \label{tab:rescustomascertain}
\end{table}
\begin{figure}[h]
	\begin{center}
		\includegraphics[width=0.95\textwidth]{img/graphs/customascertain_final_metrics.png}
		\caption{Risultati su ASCERTAIN con soggetti fittizi}
		\label{fig:customascertain_metrics_graph}
	\end{center}
\end{figure}

Dalla tabella \ref{tab:rescustomascertain} si evince subito come il bilanciamento abbia prodotto accuratezze medie molto superiori al dataset originale: l'approccio offline raggiunge l'87,77\% di accuratezza, paragonabile ai risultati ottenuti sul dataset WESAD, e gli approcci continui, in linea coi precedenti risultati, superano comunque l'80\% di accuratezza media in tutti i casi.\\
Dalle rimanenti metriche si nota comunque la difficoltà intrinseca del dataset, che raggiunge un'accuratezza finale del solo 33\% con l'approccio di replay episodico e BWT spesso negativo o comunque molto basso.
\begin{figure}[!tbp]
    \begin{minipage}[b]{0.5\textwidth}
		\includegraphics[width=\textwidth]{img/graphs/ascertain_final_accuracy.png}
		\caption{Accuratezza sul dataset ASCERTAIN originale}
		\label{fig:ascertain_accuracy_graph}
	\end{minipage}
    \hfill
    \begin{minipage}[b]{0.5\textwidth}
		\includegraphics[width=\textwidth]{img/graphs/customascertain_final_accuracy.png}
		\caption{Accuratezza su ASCERTAIN con soggetti fittizi}
		\label{fig:customascertain_accuracy_graph}
	\end{minipage}
\end{figure}
Questa variazione del dataset ASCERTAIN dimostra come il bilanciamento dei dati di training sia fondamentale nel corretto addestramento di una rete neurale. L'avere dei soggetti di partenza bilanciati a livello di classi contenute, e di conseguenza esempi di addestramento non fortemente sbilanciati su una o più classi come nel caso del dataset ASCERTAIN originale, porta ad avere un'addestramento della rete neurale di miglior qualità e risultati finali sensibilmente migliori.
\section{Approcci continui a confronto}
Dai risultati ottenuti e elencati nelle tabelle \ref{tab:reswesad}, \ref{tab:resascertain} e \ref{tab:rescustomascertain} possiamo mettere a confronto fra loro i diversi approcci di continual learning presi in considerazione.

\begin{table}[h]
\footnotesize
    \begin{tabular}{l|c|c|c|c|c|c|c}
        \textbf{Dataset} & \textbf{Epoche} & \textbf{Tempo} & \textbf{Acc.} & \textbf{ACC} & \textbf{BWT} & \textbf{FWT} & \textbf{Memoria}\\
        \hline
        \textbf{WESAD} & 49,14$\pm$33,67 & 947,24s & 73,13$\pm$4,02 & 0,7721 & 0,0343 & 0,5397 & 2173 Mb\\
        \textbf{ASCERTAIN} & 10,88$\pm$5,01 & 249,28s & 37,45$\pm$5,01 & 0,25 & -0,0168 & 0,0213 & 2154,36 Mb\\
        \textbf{Cust. ASC.} & 7,75$\pm$5,33 & 226,24s & 87,01$\pm$0,64 & 0,2481 & -0,0241 & 0,0052 & 2143,71 Mb\\
    \end{tabular}
    \caption{Risultati dell'approccio continual}
    \label{tab:rescontinual}
\end{table}

L'approccio continuativo più semplice ottiene sui dataset risultati sempre comparabili all'approccio offline utilizzato come baseline. Come elencato in tabella \ref{tab:rescontinual}, l'approccio continual su dataset WESAD ottiene un'accuratezza media del 73.13\%, da confrontare all'approccio offline con un'accuratezza del 99.07\%. Questo primo risultato porta ad una considerazione: su dataset come WESAD, cioè contenenti dati "facili", già solo l'approccio continuativo più triviale ottiene risultati non ottimi ma comunque buoni, anche dimostrato dalla metrica BWT di 0,0343 e dall'FWT di 0,5397 che indicano un buon trasferimento della conoscenza.\\
ASCERTAIN d'altro canto ottiene solo un risultato del 42.78\% offline, ma anche qua l'approccio continual ottiene un paragonabile 37.45\% di accuratezza media, nonostante l'accuratezza finale solo del 25\% (che su quattro classi da inferire è equivalente al tirare a indovinare). ASCERTAIN con i soggetti fittizi, invece, ottiene un 87.01\% di accuratezza sull'approccio continual a fronte dell'87.77\% ottenuto dall'approccio offline, dimostrando l'importanza delle classi bilanciate, ma ottenendo solo il 24.81\% di accuratezza finale.

\begin{table}[h]
\footnotesize
    \begin{tabular}{l|c|c|c|c|c|c|c}
        \textbf{Dataset} & \textbf{Epoche} & \textbf{Tempo} & \textbf{Acc.} & \textbf{ACC} & \textbf{BWT} & \textbf{FWT} & \textbf{Memoria}\\
        \hline
        \textbf{WESAD} & 39,71$\pm$19,91 & 2786s & 81,97$\pm$8,67 & 0,961 & 0,1383 & 0,4674 & 2291,45 Mb\\
        \textbf{ASCERTAIN} & 9,25$\pm$6,81 & 932,56s & 39,06$\pm$4,26 & 0,2697 & 0,0064 & 0,0278 & 2297,17 Mb\\
        \textbf{Cust. ASC.} & 7,25$\pm$8,80 & 872,8s & 87,66$\pm$0,16 & 0,2773 & 0,0032 & 0,0801 & 2273,07 Mb\\
    \end{tabular}
    \caption{Risultati dell'approccio cumulative}
    \label{tab:rescumulative}
\end{table}

L'approccio cumulativo, i cui risultati sono elencati nella tabella \ref{tab:rescumulative}, evidenzia fin da subito l'effetto che il replay degli esempi ha sull'addestramento continuativo.\\
Sul dataset WESAD, l'approccio cumulativo raggiunge un'accuratezza media dell'81.97\% e finale del 96.10\%, cioè quasi a livello con l'approccio offline che ottiene un'accuratezza del 99.07\%. Anche BWT e FWT evidenziano un buon trasferimento della conoscenza, attestandosi rispettivamente su 0.1383 e 0.4674.\\
Il dataset ASCERTAIN evidenzia ancora una volta la propria difficoltà, con un'accuratezza media dell'approccio cumulativo pari a 39.06\% e finale del 26.97\%, comunque molto ridotta rispetto al 42.78\% dell'addestramento offline. Con i soggetti bilanciati, otteniamo un'accuratezza dell'87.66\%, da comparare al quasi identico 87.77\% di accuratezza dell'approccio offline, ma un'accuratezza finale del 27.73\%. In entrambi i casi, il BWT indica un trasferimento all'indietro della conoscenza quasi nullo, rispettivamente dello 0.0064 e dello 0.0032, e analogamente l'FWT, dello 0.0278 e dello 0.0801 rispettivamente.

\begin{table}[h]
\footnotesize
    \begin{tabular}{l|c|c|c|c|c|c|c}
        \textbf{Dataset} & \textbf{Epoche} & \textbf{Tempo} & \textbf{Acc.} & \textbf{ACC} & \textbf{BWT} & \textbf{FWT} & \textbf{Memoria}\\
        \hline
        \textbf{WESAD} & 41,14$\pm$21,19 & 1063,51s & 78,29$\pm$3,32 & 0,7849 & -0,002 & 0,4582 & 2184,77 Mb\\
        \textbf{ASCERTAIN} & 13,25$\pm$10,03 & 402,96s & 39,48$\pm$4,37 & 0,2603 & -0,014 & 0,0485 & 2173,95 Mb\\
        \textbf{Cust. ASC.} & 12,125$\pm$12,94 & 299,52s & 87,13$\pm$0,75 & 0,2711 & -0,0012 & 0,306 & 2146,35 Mb\\
    \end{tabular}
    \caption{Risultati dell'approccio con il 25\% di replay}
    \label{tab:resreplay}
\end{table}
Comparandolo al 99.07\% dell'approccio offline, sul dataset WESAD avere il 25\% di replay fa ottenere risultati ovviamente inferiori all'81.97\% dell'approccio continuativo ma consente comunque di superare il 73.13\% dell'approccio continual, ottenendo un 78.29\% di accurateza media e un 78.49\% di accuratezza finale. Nonostante il BWT di -0.002, quasi nullo, si ha un buon trasferimento in avanti della conoscenza come provato dall'FWT di 0.4582.\\
ASCERTAIN ha un comportamento simile: l'approccio che usa il il 25\% di replay ottiene il 39.48\% di accuratezza media e il 26.03\% di accuratezza finale che supera l'approccio continual e anche quello cumulativo. Lo stesso non si può dire di ASCERTAIN con i soggetti fittizi, che come per WESAD ottiene un'accuratezza sul 25\% di replay che si attesta a metà tra l'approccio continual e quello cumulativo. I risultati sono elencati nella tabella \ref{tab:resreplay}

% offline
% WESAD: 28 epoche, 94.69s, 99,07%, 2061.40 Mb
% ASCERTAIN: 3 epoche, 29.14s, 42.78%, 1817.28 Mb
% CUSTOM ASCERTAIN: 29 epoche, 66.17s, 87.77%, 1945.91 Mb
\begin{table}[h]
\footnotesize
    \begin{tabular}{l|c|c|c|c|c|c|c}
        \textbf{Dataset} & \textbf{Epoche} & \textbf{Tempo} & \textbf{Acc.} & \textbf{ACC} & \textbf{BWT} & \textbf{FWT} & \textbf{Memoria}\\
        \hline
        \textbf{WESAD} & 35$\pm$26,26 & 1088,29s & 82,13$\pm$6,60 & 0,9095 & 0,0841 & 0,4226 & 2097,47 Mb\\
        \textbf{ASCERTAIN} & 12,62$\pm$7,94 & 498,24s & 38,80$\pm$4,04 & 0,2742 & 0,0048 & 0,0156 & 2231,42 Mb\\
        \textbf{Cust. ASC.} & 5,5$\pm$3,94 & 319,76s & 82,19$\pm$3,95 & 0,339 & 0,0346 & 0,1043 & 2204,84 Mb\\
    \end{tabular}
    \caption{Risultati dell'approccio episodico con $m = 70$}
    \label{tab:resepisodic}
\end{table}
% episodico
Utilizzando un replay statico e bilanciato nelle classi, vale a dire l'approccio episodico i cui risultati sono elencati in tabella \ref{tab:resepisodic}, i risultati su WESAD sono comparabili all'approccio cumulativo pur usando quasi il 10\% in meno di memoria: l'approccio episodico raggiunge l'82.13\% di accuratezza media mentre quello cumulativo l'81.97\%, ottenendo però un'accuratezza finale del 96.10\% rispetto al 90.95\% dell'approccio episodico.\\
Sul dataset ASCERTAIN abbiamo una situazione simile: l'approccio cumulativo ottiene il 38.80\% di accuratezza mentre con l'episodico si raggiunge il 39.06\% usando il 3\% in meno di memoria. Lo stesso discorso non si può dire per il dataset ASCERTAIN con i soggetti fittizi, che ottiene un'accuratezza media dell'82.19\% inferiore all'87.66\% dell'approccio cumulativo.

\begin{table}[h]
\footnotesize
    \begin{tabular}{l|c|c|c|c|c|c|c}
        \textbf{Dataset} & \textbf{Epoche} & \textbf{Tempo} & \textbf{Acc.} & \textbf{ACC} & \textbf{BWT} & \textbf{FWT} & \textbf{Memoria}\\
        \hline
        \textbf{WESAD} & 29,71$\pm$17,38 & 1342,81s & 70,74$\pm$4,74 & 0,7251 & 0,0113 & 0,4698 & 2187,40 Mb\\
        \textbf{ASCERTAIN} & 24,14$\pm$16,94 & 810,96s & 36,08$\pm$5,66 & 0,2497 & -0,0183 & -0,0003 & 2171,62 Mb\\
        \textbf{Cust. ASC.} & 14,38$\pm$7,58 & 541,68s & 87,34$\pm$0,42 & 0,2467 & -0,0191 & 0,0845 & 2146,36 Mb\\
    \end{tabular}
    \caption{Risultati dell'approccio EWC}
    \label{tab:resewc}
\end{table}
% EWC
Il primo dei due metodi provati che non si basano sui replay, l'Elastic Weight Consolidation nei risultati in tabella \ref{tab:resewc}, presenta su WESAD un'accuratezza inferiore anche all'approccio continual triviale: quest'ultimo otteneva un'accuratezza media del 73.13\% con un'accuratezza finale del 77.21\%, mentre EWC raggiunge l'accuratezza media del 70.74\% con un'accuratezza finale del 72.51\%. Nonostante questo, il trasferimento di conoscenza è paragonabile, seppur inferiore, all'approccio continual triviale.\\
Su ASCERTAIN il risultato presenta le medesime caratteristiche: con un'accuratezza media del 36.08\% e un'accuratezza finale del 24.97\% risulta di poco inferiore all'approccio continual triviale. Discorso diverso per ASCERTAIN con i soggetti fittizi, che riesce a ottenere un'accuratezza finale leggermente superiore all'approccio continual triviale con 87.34\% rispetto al 87.01\% e anhe metriche BWT e FWT superiori.

\begin{table}[h]
\footnotesize
    \begin{tabular}{l|c|c|c|c|c|c|c}
        \textbf{Dataset} & \textbf{Epoche} & \textbf{Tempo} & \textbf{Acc.} & \textbf{ACC} & \textbf{BWT} & \textbf{FWT} & \textbf{Memoria}\\
        \hline
        \textbf{WESAD} & 44,29$\pm$18,30 & 3282,51s & 69,09$\pm$5,82 & 0,7419 & 0,0451 & 0,3248 & 2121,31 Mb\\
        \textbf{ASCERTAIN} & 21,62$\pm$10,20 & 879,68s & 35,69$\pm$5,43 & 0,2448 & 0.0327 & 0,0156 & 2103 Mb\\
        \textbf{Cust. ASC.} & 20,75$\pm$8,42 & 771,36s & 86,91$\pm$0,89 & 0,2495 & -0,0333 & -0,146 & 2095,93 Mb\\
    \end{tabular}
    \caption{Risultati dell'approccio LWF}
    \label{tab:reslwf}
\end{table}
% LWF
Learning Without Forgetting, nei risultati in tabella \ref{tab:reslwf}, ha risultati comparabili ad EWC. Su WESAD risulta inferiore all'approccio continual triviale, con 69.09\% di accuratezza media contro il 70.74\% del continual, come per ASCERTAIN e ASCERTAIN con soggetti fitizi. Quest'ultimo in particolare risulta anche questa volta inferiore, seppur di poco, all'approccio continual. % TODO: rivedere