\chapter{Dataset}
Ai fini del progetto, il primo elemento necessario per poter valutare correttamente la sinergia continual learning -- Human State Monitoring sono dei dati raccolti riguardanti quest'ultimo. I dati dovrebbero provenire da più soggetti, per correttezza statistica, e riguardare diversi aspetti biometrici della persona: battito cardiaco, sudorazione, respirazione\ldots\\\\
I dataset selezionati ai fini di questa comparazione sono due, che seguono.
\section{WESAD}
Il dataset WESAD$^{\cite{10.1145/3242969.3242985}}$, acronimo di \textit{WEarable Stress and Affect Detection}, contiene dati raccolti da 15 soggetti durante uno studio effettuato in laboratorio riguardante i livelli di stress misurati tramite sensori biometrici e di movimento indossabili. Il dataset classifica questi dati in tre scale: neutralità, stress e divertimento.\\
I dispositivi usati per la raccolta dei dati sono due: uno indossato sul petto (RespiBAN) e uno indossato sul polso (Empatica E4).\\\\Il RespiBAN, indossato sul petto, raccoglie dati campionati a 700 Hz su:
\begin{itemize}
    \item[-] Elettrocardiogramma (ECG)
    \item[-] Attività elettrodermica (EDA)
    \item[-] Elettromiogramma (EMG)
    \item[-] Respirazione
    \item[-] Temperatura corporea
    \item[-] Accelerazione sui tre assi
\end{itemize}
\pagebreak
L'Empatica E4, indossato sul polso, monitora con frequenza di campionamento eterogenea:
\begin{itemize}
    \item[-] Pulsazioni (BVP), a 64Hz
    \item[-] Attività elettrodermica (EDA), a 4 Hz
    \item[-] Temperatura corporea, a 4 Hz
    \item[-] Accelerazione sui tre assi, a 32 Hz
\end{itemize}
La quantità di dati contenuti in questo dataset lo rende un ottimo strumento nell'addestramento di modelli di machine learning riguardanti l'ambiente dello HSM. Dati come le pulsazioni, la temperatura e l'elettrocardiogramma sono precisissimi indicatori biometrici dello stato psico-fisico di una persona e pertanto sono estremamente utili alle reti neurali per dedurre lo stato psico-fisico di soggetto.
\subsection{Preprocessing}
Una volta caricato il dataset, i dati sono stati concatenati e ricampionati a 32 Hz, attraverso la libreria \texttt{SciPy}.
\lstinputlisting[style=myPython, firstnumber=31, firstline=31, lastline=42]{code/wesad_data.py}
Dopodiché, l'insieme delle features è stato standardizzato ponendo media pari a 0 e deviazione standard uguale a 1
\lstinputlisting[style=myPython, firstnumber=46, firstline=46, lastline=46]{code/wesad_data.py}
Sono state ricampionate le etichette e sono state rimosse l'etichetta 0, corrispondente ad una condizione di neutralità, e le etichette 5, 6 e 7 poichè secondo le indicazioni del dataset sono corrispondenti a dati da ignorare.
\lstinputlisting[style=myPython, firstnumber=50, firstline=50, lastline=52]{code/wesad_data.py}
\lstinputlisting[style=myPython, firstnumber=56, firstline=56, lastline=57]{code/wesad_data.py}
\lstinputlisting[style=myPython, firstnumber=86, firstline=86, lastline=87]{code/wesad_data.py}
I dati sono poi raccolti in sottosequenze di 100 punti relativi alla stessa etichetta, corrispondenti a circa 3 secondi, e per ogni etichetta vengono estratte 100 di queste sottosequenze.
\lstinputlisting[style=myPython, firstnumber=62, firstline=62, lastline=82]{code/wesad_data.py}
Dai dati viene poi estratto un test set. Questo ci lascia con un test set di 1500 elementi e un training set da 4500 elementi suddivisi come specificato nella tabella \ref{tab:splittedwesad}.

\section{ASCERTAIN }
ASCERTAIN$^{\cite{10.1145/2818346.2820736}}$, acronimo di \textit{multimodal databASe for impliCit pERsonaliTy and Affect recognitIoN}, è un dataset contenente dati provenienti da 58 soggetti raccolti con sensori fisiologici commerciali e classifica le informazioni su diverse scale: incitamento, valenza, investimento, apprezzamento e familiarità. Il dataset contiene anche dati relativi all'attività facciale dei soggetti registrati con sensori comuni, oltre ai dati sull'elettroencefalogramma (EEG), elettrocardiogramma (ECG) e sulle risposte galvaniche della pelle (GSR).\\\\
Il motivo principale per cui è stato selezionato ASCERTAIN sono i dati sui movimenti facciali dei soggetti. In contesti reali, dove non sempre è possibile raccogliere dati sull'utente mediante sensori biometrici applicati sul corpo del soggetto, spesso è disponibile solamente una o più videocamere che lo riprendono. Poter sfruttare questi dati non invasivi per poter trarre conclusioni relative allo HSM può rivelarsi quindi molto utile in applicazioni reali.
\subsection{Preprocessing}
Dal dataset sono stati esclusi due soggetti poiché i loro dati risultavano imprecisi o incompleti. Ogni soggetto rimanente è stato caricato e per ognuno sono state create le etichette seguendo la seguente suddivisione basata sui valori di incitamento e valenza specificati, che corrispondono ai quadranti del grafico cartesiano creato dai due valori:
\begin{itemize}
    \item[-] Label 0, corrispondente a valori di incitamento $> 3$ e valenza $> 0$
    \item[-] Label 1, corrispondente a valori di incitamento $> 3$ e valenza $\leq 0$
    \item[-] Label 2, corrispondente a valori di incitamento $\leq 3$ e valenza $> 0$
    \item[-] Label 3, corrispondente a valori di incitamento $\leq 3$ e valenza $\leq 0$
\end{itemize}
\lstinputlisting[style=myPython, firstnumber=40, firstline=40, lastline=51]{code/ascertain_data2.py}
I dati sono poi ripuliti dai valori mancanti, ricampionati a 32 Hz e concatenati fra loro.
\lstinputlisting[style=myPython, firstnumber=55, firstline=55, lastline=58]{code/ascertain_data2.py}
\lstinputlisting[style=myPython, firstnumber=62, firstline=62, lastline=64]{code/ascertain_data2.py}
\lstinputlisting[style=myPython, firstnumber=68, firstline=68, lastline=76]{code/ascertain_data2.py}
Vengono poi costruite le sottosequenze, da 160 elementi ciascuna (5 secondi), mantenendo un bilanciamento fra le diverse classi.
\lstinputlisting[style=myPython, firstnumber=80, firstline=80, lastline=80]{code/ascertain_data2.py}
\lstinputlisting[style=myPython, firstnumber=82, firstline=82, lastline=104]{code/ascertain_data2.py}
Come per WESAD, anche da ASCERTAIN viene estratto un test set, il che ci lascia con un test set da 720 elementi e un training set da 2160 elementi suddivisi come indicato nella tabella \ref{tab:splittedascertain}

\begin{table}[h]
    \parbox{.45\linewidth}{
    	\begin{center}
    		\begin{tabular}{l|c}
    		     \textbf{Soggetto} & \textbf{Dati}\\
    		     \hline
    		     S2 & 287 \\
    		     S3 & 287 \\
    		     S4 & 298 \\
    		     S5 & 298 \\
    		     S6 & 297 \\
    		     S7 & 303 \\
    		     S8 & 309 \\
    		     S9 & 292 \\
    		     S10 & 313 \\
    		     S11 & 292 \\
    		     S13 & 308 \\
    		     S14 & 297 \\
    		     S15 & 305 \\
    		     S16 & 313 \\
    		     S17 & 301 \\
    		     \hline
    		     Totale & 4500
    		\end{tabular}
    		\caption{Datataset WESAD}
    		\label{tab:splittedwesad}
    	\end{center}
	}
    \parbox{.45\linewidth}{
    	\begin{center}
    		\begin{tabular}{l|c}
    		     \textbf{Soggetto} & \textbf{Dati}\\
    		     \hline
    		     S0 & 95 \\
    		     S1 & 75 \\
    		     S2 & 148 \\
    		     S3 & 154 \\
    		     S4 & 153 \\
    		     S5 & 208 \\
    		     S6 & 131 \\
    		     S7 & 115 \\
    		     S8 & 119 \\
    		     S9 & 206 \\
    		     S10 & 119 \\
    		     S11 & 78 \\
    		     S12 & 74 \\
    		     S13 & 109 \\
    		     S14 & 72 \\
    		     S15 & 84 \\
    		     S16 & 220 \\
    		     \hline
    		     Totale & 2160
    		\end{tabular}
    		\caption{Datataset ASCERTAIN}
    		\label{tab:splittedascertain}
    	\end{center}
    }
\end{table}
% aggiungere queste parti? \/
% presi in considerazione diversi (WESAD, HHAR (\cite{10.1145/2809695.2809718}) per attività fisiche, PAMAP2 (\cite{6246152}) per dati cuore e movimento, OPPORTUNITY (\cite{5573462}) per movimenti del corpo, ASCERTAIN)
% WESAD e ASCERTAIN scelti principalmente per i dati biometrici contenuti, ASCERTAIN (\cite{10.1145/2818346.2820736}) in particolare dati facciali utili in contesti reali dove non è sempre possibile attaccare sensori all'utente, mentre WESAD concentrato principalmente sullo stress quindi utile in altri contesti