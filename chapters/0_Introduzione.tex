\phantomsection
\chapter*{Introduzione}
\markboth{INTRODUZIONE}{}
\addcontentsline{toc}{chapter}{Introduzione}

Negli ultimi decenni, le "intelligenze artificiali" si sono fatte sempre più spazio nella nostra vita di tutti i giorni. Grazie ai nostri smartphone, usiamo modelli di machine learning ultra efficienti che indirizzano le nostre ricerche nella rete, e grazie a tali modelli le aziende possono mostrarci inserzioni sempre più aderenti ai nostri gusti personali. Ma il mondo del machine learning ha importanti e utili applicazioni anche nella salvaguardia del benessere dell'individuo, ad esempio in contesti come il riconoscimento delle malattie o l'interpretazione del linguaggio parlato. Questi contesti, che rientrano nell'ambito dello \textit{human state monitoring}, stanno diventando sempre più importanti nella vita di tutti i giorni e diventerà sempre più necessario l'avere a disposizione sistemi che sappiano sfruttare efficacemente i dati sull'utente che hanno a disposizione, così da realizzare sistemi responsivi e reattivi in base allo stato attuale dell'utente.\\\\
Attualmente, diversi studi concentrano i propri sforzi sul risolvere alcune problematiche intrinseche nell'apprendimento continuo, come il \textit{catastrophic forgetting}$^{\cite{kirkpatrick2017overcoming},\cite{li2017learning},\cite{DBLP:journals/corr/Lopez-PazR17},\cite{rolnick2019experience}}$ o riguardanti la realizzazione in contesti reali di produzione di infrastrutture di continual learning$^{\cite{diethe2019continual}}$, mentre ulteriori studi hanno analizzato l'applicabilità in altri contesti come la \textit{human activity recognitio}$^{\cite{Jha_2021}}$ o l'acquisizione di informazioni dai social network$^{\cite{priyanshu2021continual}}$. Sono rari gli studi sullo \textit{human state recognition} e ancora non è stato realizzato un confronto fra tecniche di apprendimento continuo su questa tipologia di dati.\\
Il nostro obiettivo è dunque mettere a confronto alcune metodologie di addestramento continual su dati relativi allo \textit{human state monitoring}.
\pagebreak
\subsection*{Obiettivi}
Gli obiettivi di questo progetto sono i seguenti:
\begin{itemize}
    \item[-] \textbf{Individuare dataset contenenti dati realitivi allo \textit{human state monitoring}}\\
    Per poter valutare la sinergia tra continual learning e \textit{human state monitoring}, inanzitutto si rende necessaria la raccolta di dati riguardanti lo HSM. Ne sono stati individuati due, scelti per le caratteristiche che verranno discusse in seguito: WESAD$^{\cite{10.1145/3242969.3242985}}$ e ASCERTAIN$^{\cite{7736040}}$.
    \item[-] \textbf{Selezionare le metriche e le informazioni utili alla comparazione dei vari approcci}\\
    Comparare i vari approcci significa avere a disposizione per ognuno di essi delle metriche che vadano a misurare i loro vari aspetti, come il tempo necessario all'addestramento, l'accuratezza raggiunta, la conoscenza trasferita e l'impatto a livello di memoria.
    \item[-] \textbf{Scegliere una \textit{baseline}}\\
    Cioè un approccio base da cui partire e che sia utile a confrontare tutti gli altri. Come sarà specificato in seguito, l'approccio scelto come baseline sarà quello \textit{offline}: ciò consiste nell'addestramento di una rete neurale usando l'intero training set a disposizione senza usare approcci continual.
    \item[-] \textbf{Raccogliere i dati necessari}\\
    Si rende necessario individuare un ambiente su cui poter eseguire le computazioni necessarie all'addestramento, trovare le API e i linguaggi di programmazione adatti allo scopo e produrre del codice verificabile e degli esperimenti replicabili.
    \item[-] \textbf{Confrontare i risultati e trarre le conclusioni}\\
    Una volta raccolti tutti i dati necessari, essi verranno confrontati per poter finalmente valutare la sinergia fra continual learning e human state monitoring.
\end{itemize}

\subsection*{Risultati}
Dagli esperimenti presentati in questa tesi si evince quanto le tecniche di apprendimento continuo siano applicabili a contesti di \textit{human state monitoring} con risultati anche comparabili all'apprendimento classico \textit{offline}. In particolare, molte delle tecniche di addestramento continuo adottate riescono a ottenere risultati inferiori ma comunque buoni quando applicate alla medesima struttura di rete neurale utilizzata nell'addestramento offline, con misure di trasferimento della conoscenza spesso anche molto positive.

\iffalse
\subsection*{Conclusione} Lo \textit{human state monitoring} è un ambiente sempre più utile e importante nella realizzazione di sistemi personalizzati che rispondono dinamicamente allo stato attuale dell'utente: un esempio può essere la riduzione della velocità di un veicolo a guida autonoma quando l'utente alla guida è in stato di sonnolenza, e quindi poco attento agli eventuali errori del veicolo.\\
Per realizzare questi modelli di machine learning si rendono necessarie reti neurali ricorrenti, poiché abbiamo a che fare con flussi di dati temporali raccolti da vari sensori posti sull'utente, che devono adattarsi in maniera continuativa all'utente attuale (nell'esempio, il proprietario del veicolo) avvalendosi quindi di tecniche di continual learning, che non soffrano del \textit{catastrophic forgetting}.\\
Si necessitano quindi di metriche per misurare nel tempo l'efficienza e la precisione di questi modelli sottoposti a continual learning, per giudicare l'applicabilità delle tecniche in esame applicate allo \textit{human state monitoring}.
\fi