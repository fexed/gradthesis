\chapter{Obiettivi}
In un mondo sempre più interconnesso e in rapida evoluzione, approcci statici al machine learning possono diventare impraticabili quando si tratta di produrre sistemi in grado di adattarsi dinamicamente allo stato fisico e mentale degli utenti che vi interagiscono. Inoltre, con l'ampia presenza di dispositivi in grado di raccogliere anche in tempo reale informazioni sulla condizione attuale degli utenti, diventano continuamente disponibili nuovi dati con cui perfezionare i modelli di machine learning in uso attraverso tecniche di continual learning.
\section{Stato dell'Arte}
Alcuni recenti studi nell'ambito del continual learning si sono concentrati, ad esempio, sull'applicabilità pratica e l'ideazione di un'architettura in grado di manutenere modelli di machine learning in produzione e gestire la dinamicità e i cambiamenti sui dati$^{\cite{diethe2019continual}}$, oppure in ambiti come la \textit{human activty recognition} (HAR)$^{\cite{Jha_2021}}$, cioè il riconoscimento delle attività quotidiane delle persone attraverso dei sensori, o anche l'acquisizione di informazioni dai social media per, ad esempio, gestire al meglio situazioni di crisi$^{\cite{priyanshu2021continual}}$.