\chapter{Conclusioni}
Dai risultati raccolti e presentati nel precedente capitolo è evidente la differenza di accuratezza raggiunta sui due dataset. In particolare, nonostante si riesca facilmente ad ottenere un modello con risultati buoni o ottimi sul dataset WESAD, ciò non è stato ottenuto sul dataset ASCERTAIN originale mentre si hanno risultati sensibilmente migliori sullo stesso dataset ASCERTAIN modificato rendendo ogni soggetto più bilanciato riguardo le classi contenute.\\\\
In tutti i dataset, le metodologie continue hanno mostrato accuratezza media inferiore rispetto all'addestramento offline: sul dataset WESAD si ha una riduzione dell'accuratezza media del 23.17\% in media sui vari approcci, mentre sul dataset ASCERTAIN si attesta sul 5.02\%. Su ASCERTAIN modificato rendendo ogni soggetto più bilanciato, invece, la riduzione media è dell'1.39\%, con la maggior parte delle metodologie che ottengono un'accuratezza inferiore di meno dell'1\%.\\
Questa riduzione dell'accuratezza è sicuramente mitigabile attraverso la selezione di un modello con performance migliori delle metodologie continue, invece di eseguire la \textit{model selection} sulla base dei risultati ottenuti nella metodologia offline. Con questi dati però si può già affermare che nella maggior parte dei casi le metodologie continue sono paragonabili all'addestramento classico, posto di avere un dataset con determinate caratteristiche: le classi all'interno del dataset di addestramento devono essere bilanciate fra loro, o si avrà un aumento di difficoltà nell'addestramento della rete neurale.\\\\
\pagebreak

Si può quindi concludere che su certe tipologie di dati, il continual learning è applicabile all'ambito dello \textit{human state monitoring} con un elevato grado di accuratezza e ottenendo modelli di machine learning in grado di ottenere risultati comparabili o simili alla metodologia offline. Il risultato più vicino all'addestramento offline è stato ottenuto con la metodologia cumulativa su dataset WESAD, con il replay al 25\% sul dataset ASCERTAIN e di nuovo con l'addestramento cumulativo sul dataset ASCERTAIN con i soggetti bilanciati artificialmente. Questo dimostra che mantenere esempi precedenti porta ad un addestramento di maggior qualità e un modello più capace di inferire correttamente lo stato psico-fisico di una persona, posto di avere sufficiente memoria per lo stoccaggio dei dati di replay. Una possibile soluzione al problema della memorizzazione potrebbe essere addestrare un modello generativo, ovvero una rete neurale che fornita una classe del dataset genera delle feature che corrispondono a quella classe. Così facendo si può realizzare un replay generativo, artificiale, che non richiede la memorizzazione di istanze ma in compenso rende necessario l'addestramento di una seconda rete neurale totalmente diversa e le cui performance hanno diretto impatto sui risultati del modello predittivo.