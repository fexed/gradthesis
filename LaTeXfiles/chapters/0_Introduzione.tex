\phantomsection
\chapter*{Introduzione}
\markboth{INTRODUZIONE}{}
\addcontentsline{toc}{chapter}{Introduzione}

Negli ultimi decenni, l'"Intelligenza Artificiale" se è fatta sempre più spazio nella nostra vita di tutti i giorni. Grazie ai nostri smartphone, usiamo modelli di Machine Learning efficienti che indirizzano le nostre ricerche nella rete, e grazie a tali modelli le aziende possono mostrarci inserzioni sempre più aderenti ai nostri gusti personali. Ma il mondo del Machine Learning ha importanti e utili applicazioni anche nella salvaguardia del benessere dell'individuo, ad esempio in contesti come il riconoscimento delle malattie o l'interpretazione del linguaggio parlato. Questi contesti, che rientrano nell'ambito dello \textit{Human State Monitoring}, stanno diventando sempre più importanti nella vita di tutti i giorni e diventerà sempre più necessario l'avere a disposizione sistemi che sappiano sfruttare efficacemente i dati sull'utente che hanno a disposizione, così da realizzare sistemi responsivi e reattivi in base allo stato attuale dell'utente.\\\\
Con Addestramento Continuo si definisce l'approccio all'addestrare un modello di Machine Learning in diverse sessioni di addestramento distribuite nel tempo, con l'obiettivo di migliorare gradualmente la sua performance. Questo approccio può portare ad un graduale deterioramento delle conoscenze acquisite nelle sessioni di addestramento iniziali: la nuova conoscenza può soppiantare la conoscenza già appresa, e questo fenomeno prende il nome di \textit{catastrophic forgetting}. Attualmente, diversi studi concentrano i propri sforzi sul risolvere alcune problematiche intrinseche nell'Addestramento Continuo, come il \textit{catastrophic forgetting}$^{\cite{kirkpatrick2017overcoming},\cite{li2017learning},\cite{DBLP:journals/corr/Lopez-PazR17},\cite{rolnick2019experience}}$ o riguardanti la realizzazione in contesti reali di produzione di infrastrutture di continual learning$^{\cite{diethe2019continual}}$, mentre ulteriori studi hanno analizzato l'applicabilità in altri contesti come la \textit{Human Activity Recognition}$^{\cite{Jha_2021}}$ o l'acquisizione di informazioni dai social network$^{\cite{priyanshu2021continual}}$. Sono rari gli studi sullo \textit{Human State Monitoring} e ancora non è stato realizzato un confronto fra le tecniche di Addestramento Continuo di questa tesi.\\
L'obiettivo è dunque mettere a confronto alcune metodologie di Addestramento Continuo su dati relativi allo \textit{Human State Monitoring}.
\subsection*{Obiettivi}
L'obiettivo principale di questa tesi è studiare, attraverso una serie di esperimenti, le performance raggiunte da alcune metodologie di Apprendimento Automatico Continuo applicate a Reti Neurali Ricorrenti riguardo problemi che rientrano nell'ambito dello \textit{Human State Monitoring}. Per poter realizzare questo obiettivo sono stati individuati alcuni sotto-obiettivi necessari alla realizzazione degli esperimenti e della comparazione.
\begin{itemize}
    \item[-] \textbf{Individuare dataset contenenti dati realitivi allo \textit{Human State Monitoring}}\\
    Per poter valutare la sinergia tra continual learning e \textit{Human State Monitoring}, inanzitutto si rende necessaria la raccolta di dati riguardanti lo HSM. Ne sono stati individuati due, scelti per le caratteristiche che verranno discusse in seguito: WESAD$^{\cite{10.1145/3242969.3242985}}$ e ASCERTAIN$^{\cite{7736040}}$.
    \item[-] \textbf{Selezionare le metriche e le informazioni utili alla comparazione delle varie metodologie}\\
    Comparare le varie metodologie significa avere a disposizione per ognuna di esse delle metriche che vadano a misurare i loro vari aspetti, come il tempo necessario all'addestramento, l'accuratezza raggiunta, la conoscenza trasferita e l'impatto a livello di memoria.
    \item[-] \textbf{Scegliere una \textit{baseline}}\\
    Cioè una metodologia base da cui partire e che sia utile a confrontare tutti gli altri. Come sarà specificato in seguito, la metodologia scelta come baseline sarà quella \textit{offline}: ciò consiste nell'addestramento di una rete neurale usando l'intero training set a disposizione senza usare approcci continual.
    \item[-] \textbf{Definire gli strumenti tecnologici}\\
    Si rende necessario individuare un ambiente in cui poter eseguire le computazioni necessarie all'addestramento, identificare le API e i linguaggi di programmazione adatti allo scopo e produrre del codice verificabile e degli esperimenti replicabili.
    \item[-] \textbf{Confrontare i risultati e trarre le conclusioni}\\
    Una volta raccolti tutti i dati necessari, essi verranno confrontati per poter finalmente valutare la sinergia fra continual learning e Human State Monitoring.
\end{itemize}
\pagebreak
\subsection*{Risultati}
Gli esperimenti e i risultati prodotti nell'ambito di questa tesi mostreranno quanto le tecniche di Addestramento Continuo siano applicabili a contesti di \textit{Human State Monitoring} con risultati anche comparabili all'apprendimento classico \textit{offline}. In particolare, molte delle tecniche di addestramento continuo adottate riescono a ottenere risultati inferiori ma comunque buoni quando applicate alla medesima struttura di rete neurale utilizzata nell'addestramento offline, con misure di trasferimento della conoscenza spesso anche molto positive.

\subsection*{TEACHING}
Questa tesi è stata svolta nell'ambito del progetto TEACHING$^{\cite{teaching2020}}$.\\TEACHING è un progetto finanziato dall'Unione Europea volto a progettare una piattaforma informatica ed il relativo toolkit software a supporto dello sviluppo e del rilascio di applicazioni autonome, adattive e affidabili, consentendo a tali applicazioni di sfruttare il feedback umano per guidare, ottimizzare e personalizzare i servizi offerti.

\subsection*{Struttura della tesi}
Nel capitolo 1 verranno definite nel dettaglio le conoscenze preliminari necessarie per affrontare gli argomenti di questa tesi. Nei due capitoli successivi verranno presentati, rispettivamente, le due raccolte di dati utilizzate negli esperimenti (capitolo 2) e gli approcci di addestramento continuo messe a confronto (capitolo 3).
Nel capitolo 4 vengono presentati nel dettaglio i risultati prodotti nell'ambito della tesi. Infine, nel capitolo 5, sono delineate le conclusioni tratte a partire da tali risultati.